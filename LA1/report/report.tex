\documentclass[a4paper, oneside]{article}
\usepackage[english, russian]{babel}

\usepackage{fontspec}
\setmainfont[
  Ligatures=TeX,
  Extension=.otf,
  BoldFont=cmunbx,
  ItalicFont=cmunti,
  BoldItalicFont=cmunbi,
]{cmunrm}
\usepackage{unicode-math}

\usepackage[bookmarks=false]{hyperref}
\hypersetup{pdfstartview={FitH},
            pdfauthor={Павел Соболев}}

\usepackage[lmargin=23mm]{geometry}

\usepackage[table]{xcolor}
\usepackage{booktabs}
\usepackage{caption}

\usepackage{graphicx}
\graphicspath{ {./figures/} }

\usepackage{sectsty}
\sectionfont{\centering}
\subsubsectionfont{\centering \vspace{-0.5em}\normalfont\itshape}

\newcommand{\npar}{\par\vspace{\baselineskip}}
\newcommand{\su}{\vspace{-0.5em}}
\newcommand{\sd}{\vspace{0.5em}}

\setlength{\parindent}{0pt}

\hypersetup{pdftitle={Астрофизическая практика: отчет по первой работе}}
\usepackage{makecell}

\begin{document}

\section*{Работа №1: Классификация звёздных спектров}
\subsubsection*{Выполнил: Павел Соболев}

\vspace{3em}

\subsection*{Задачи}

\begin{itemize}
  \setlength\itemsep{-0.1em}
  \item Получить спектр звезды, используя моделируемые с помощью компьютера телескоп и спектрограф;
  \item Сравнить этот спектр со спектром звезды известного спектрального класса;
  \item Отождествить абсорбционные линии на графическом и фотографическом изображениях спектра;
  \item Обсудить относительную глубину абсорбционных линий, измерив их, и сравнить со стандартным спектром;
\end{itemize}

\subsection*{Ход выполнения и результаты}

В ходе работы с виртуальным телескопом и спектрографом были получены следующие данные.
Отождествления проведены по наличию линий, их интенсивности, а также по общим характеристикам спектра.
Пояснения приведены в примечаниях.

\begin{table}[h]
  \centering
  \caption{Данные и результаты отождествлений (часть 1)}
  \begin{tabular}{ccccc}
    \toprule
    № & Зв. величина & Класс & Линии ($\lambda$ в \AA, интенсивность) & Примечания \\
    \midrule

    78 & 11.146 & G2 IV &
    \begin{tabular}{@{}c@{}}
      K Ca II (3933.68, 0.150) \\
      H Ca II (3968.49, 0.160) \\
      H$_\varepsilon$ (3970.07, 0.180) \\
      Fe I (4045.82, 0.700) \\
      H$_\delta$ (4101.75, 0.660) \\
      Ca I (4226.74, 0.700) \\
      G Band (4300.00, 0.480)
    \end{tabular} &
    \makecell{
      Интенсивные \\
      линии H и K Ca II, \\
      ослабленная линия H I, \\
      есть линия Ca I и \\
      многочисленные \\
      линии металлов
    } \\

    \arrayrulecolor{black!40}
    \midrule

    140 & 10.086 & B6 V &
    \begin{tabular}{@{}c@{}}
      H Ca II (3968.49, 0.410) \\
      H$_\varepsilon$ (3970.07, 0.400) \\
      He II (4100.04, 0.390) \\
      H$_\delta$ (4101.60, 0.400) \\
      H$_\gamma$ (4340.48, 0.330) \\
      Fe I (4143.88, 0.770)
    \end{tabular} &
    \makecell{
      Линии поглощения \\
      водорода и гелия, \\
      сильная линия H Ca II
    } \\

    \midrule

    142 & 7.572 & M2 I &
    \begin{tabular}{@{}c@{}}
      Ca I (4226.74, 0.130) \\
      H Ca II (3968.49, 0.070) \\
      Mn I (4030.76, 0.180) \\
      Fe I (4383.56, 0.430) \\
      G Band (4300.00, 0.360)
    \end{tabular} &
    \makecell{
      Изреженный спектр, \\
      линии молекул, \\
      слабая полоса G
    } \\

    \arrayrulecolor{black}
    \bottomrule
  \end{tabular}
\end{table}

\newpage

\begin{table}[h]
  \centering
  \caption{Данные и результаты отождествлений (часть 2)}
  \begin{tabular}{ccccc}
    \toprule
    № & Зв. величина & Класс & Линии ($\lambda$ в \AA, интенсивность) & Примечания \\
    \midrule

    433 & 5.658 & O8 V &
    \begin{tabular}{@{}c@{}}
      H Ca II (3968.49, 0.670) \\
      He I (4024.80, 0.820) \\
      He II (4100.04, 0.700) \\
      H$_\gamma$ (4340.48, 0.530)
    \end{tabular} &
    \makecell{
      Слабые линии \\
      нейтрального водорода, \\
      гелия, ионизованного гелия
    } \\

    \arrayrulecolor{black!40}
    \midrule

    146 & 10.596 & F0 V &
    \begin{tabular}{@{}c@{}}
      K Ca II (3933.68, 0.420) \\
      H Ca II (3968.49, 0.320) \\
      H$_\gamma$ (4340.48, 0.440) \\
      Fe I (4383.56, 0.790) \\
      Ca I (4226.74, 0.820) \\
      G Band (4300.00, 0.770)
    \end{tabular} &
    \makecell{
      Усиленные линии \\
      H и K Ca II, слабые \\
      линии металлов, \\
      появляется линия Ca I, \\
      появляется полоса G
    } \\

    \midrule

    152 & 11.282 & B6 V &
    \begin{tabular}{@{}c@{}}
      H Ca II (3968.49, 0.390) \\
      He I (4120.82, 0.740) \\
      He II (4100.04, 0.380) \\
      H$_\delta$ (4101.75, 0.390) \\
      H$_\gamma$ (4340.48, 0.330)
    \end{tabular} &
    \makecell{
      Линии поглощения \\
      водорода и гелия, \\
      сильная линия H Ca II
    } \\

    \midrule

    155 & 8.330 & K0 V &
    \begin{tabular}{@{}c@{}}
      K Ca II (3933.68, 0.260) \\
      G Band (4300.00, 0.500) \\
      Fe I (4383.56, 0.620) \\
      Ca I (4226.74, 0.550) \\
      H$_\delta$ (4101.75, 0.650)
    \end{tabular} &
    \makecell{
      Линии металлов и \\
      полоса G интенсивны, \\
      слабые линии водорода
    } \\

    \midrule

    68 & 7.666 & K0 III &
    \begin{tabular}{@{}c@{}}
      K Ca II (3933.68, 0.060) \\
      H Ca II (3968.48, 0.110) \\
      H$_\varepsilon$ (3970.07, 0.510) \\
      Mn I (4030.76, 0.510) \\
      Fe I (4045.82, 0.420) \\
      G Band (4300.00, 0.410) \\
      Fe I (4283.56, 0.530) \\
    \end{tabular} &
    \makecell{
      Линии металлов и \\
      полоса G интенсивны, \\
      слабые линии водорода
    } \\

    \midrule

    272 & 11.031 & F0 II &
    \begin{tabular}{@{}c@{}}
      K Ca II (3933.68, 0.310) \\
      H Ca II (3968.49, 0.240) \\
      H$_\gamma$ (4340.48, 0.290) \\
      Ca I (4226.74, 0.690)
    \end{tabular} &
    \makecell{
      Усиленные линии \\
      H и K Ca II, \\
      появляется линия Ca I
    } \\

    \midrule

    147 & 9.351 & M2 I &
    \begin{tabular}{@{}c@{}}
      K Ca II (3933.68, 0.030) \\
      H Ca II (3968.49, 0.050) \\
      Mn I (4030.76, 0.200) \\
      Ca I (4226.74, 0.240) \\
      G Band (4300.00, 0.340)
    \end{tabular} &
    \makecell{
      Изреженный спектр, \\
      линии молекул, слабые \\
      линии полосы G
    } \\

    \arrayrulecolor{black}
    \bottomrule
  \end{tabular}
\end{table}

\end{document}